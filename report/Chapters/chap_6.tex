\chapter{System Design}
\justify
\quad 
\begin{itemize}
    \item Data Collection: Collect the data on the number of study hours and percentage of students from various sources, such as online databases or manual data entry.
    \item Data Preprocessing: Clean and preprocess the data, which may include handling missing values, removing duplicates, and scaling the features.
    \item Model Selection: Choose the appropriate unsupervised learning algorithm for clustering the data, such as K-Means, Hierarchical Clustering, or DBSCAN.
    \item Optimal Cluster Selection: Determine the optimal number of clusters using various methods, such as the Elbow method or Silhouette method.
    \item Training the Model: Split the data into training and testing sets, and use the training set to train a model using various libraries like Scikit-learn, TensorFlow, Keras, PyTorch, or Apache Spark  library.
    \item Model Deployment: Deploy the trained model into a web application or mobile application.
    \item Model Training: Train the chosen model using the Scikit-learn library and the optimal number of clusters determined in the previous step.
    \item Model Evaluation: Evaluate the performance of the model using various evaluation metrics, such as Silhouette score or Calinski-Harabasz score.
    \item Visualization: Visualize the clustering results using various plotting libraries, such as Matplotlib or Seaborn.
\end{itemize}
\section{System Analysis}
\justify
\quad
\begin{itemize}
    \item \textbf{Identify Business Objectives:} Identify the business objectives and goals that the Data Science and Business Analytics system aims to achieve.

{Determine Data Sources:} Determine the sources of the data required and collect the data from various sources, such as online databases, manual data entry, or web scraping.









    \item \textbf{Define Data Requirements:} Identify the data required to achieve the business objectives and ensure the quality and reliability of the data.
    \item \textbf{Data Preparation:} Clean and preprocess the data, which may include handling missing values, removing duplicates, and scaling the features.
    \item \textbf{Exploratory Data Analysis (EDA):} Analyze and visualize the data to gain insights, discover patterns, and identify relationships between the features.
    \item \textbf{Feature Engineering:} Extract and transform the relevant features from the data, which may include feature scaling, normalization, or dimensionality reduction.
    \item \textbf{Model Selection and Training:} Choose the appropriate machine learning algorithm for the task at hand, and train the model using the selected algorithm and the preprocessed features.
    \item \textbf{Model Evaluation:} Evaluate the performance of the model using various evaluation metrics, such as accuracy, precision, recall, or F1-score.
    \item \textbf{Model Deployment:} Deploy the trained model into a web application or mobile application, which will take the input data from the user and provide the predicted output as per the trained model.
    \item \textbf{Business Analysis:} Analyze the impact of the model on the business objectives and identify opportunities for improvement or optimization.
    \item \textbf{Continuous Improvement: }Continuously monitor and update the model based on new data and feedback, and improve the model's performance over time.
\end{itemize}
\newpage

\section{Quality Of Service}
\justify \quad
\begin{itemize}
    \item \textbf {Accuracy:} The accuracy of the models and predictions is a crucial factor in determining the QoS. The models should be able to provide accurate results to meet the business objectives.
    \item \textbf {Speed:} The speed of the algorithms and the overall system is important in ensuring timely results. The system should be able to process the data and provide results within the required timeframe.
    \item \textbf {Scalability:} The system should be able to scale up or down depending on the volume of data and computational resources required. This ensures that the system can handle increasing workloads and provide reliable results.
    \item \textbf {Security:} Data security and privacy are critical factors in ensuring the QoS of the system. The system should have robust security features and measures in place to protect sensitive data and prevent unauthorized access.
    \item \textbf {User Interface:} The user interface and ease of use are important in ensuring the QoS of the system. The system should have a user-friendly interface that allows users to easily interact with the system and access the results
    \item \textbf {Reliability:} The system should be reliable and able to handle unexpected events or errors. It should have measures in place to prevent data loss and ensure the system can recover from failures.
\end{itemize}

\subsection{Design Consideration}
\justify \quad
Data Storage: The system should be designed to handle large volumes of data efficiently. This includes choosing the appropriate data storage technologies, such as data warehouses, data lakes, or cloud storage, and implementing data partitioning and indexing to optimize data retrieval.
\newline 
\justify\quad
Data Cleaning and Preparation: Data cleaning and preparation are important steps in data analysis. The system should include tools for data cleaning, such as removing duplicates and missing values, and data preparation, such as feature selection and transformation.
\newline 
\justify\quad
Model Selection and Training: The system should be able to accommodate various machine learning models and algorithms, and provide tools for model selection and training. The system should also have capabilities for hyperparameter tuning, cross-validation, and model evaluation.
\newline 
\justify\quad
Visualization and Reporting: The system should include tools for data visualization and reporting, such as dashboards and interactive charts, to enable users to gain insights from the data and communicate the results effectively.
\newline 
\justify\quad
Performance and Scalability: The system should be designed to handle large volumes of data and scale up or down as needed. This includes implementing parallel processing and distributed computing to improve performance and scalability.
\newline 
\justify\quad
Security and Privacy: The system should have robust security and privacy measures in place to protect sensitive data and prevent unauthorized access. This includes implementing access controls, encryption, and data anonymization.
\newline 
\justify\quad
Integration and Interoperability: The system should be able to integrate with other systems and tools, and support interoperability standards to enable data exchange and sharing.

\newpage